\begin{frame}[noframenumbering,plain]
    \setcounter{framenumber}{1}
    \begin{center}
        {\thesisOrganization}
    \end{center}
    
    \begin{center}
        \vspace{10mm}\textbf{\thesisTitle}
    \end{center}
    
    \begin{flushright}
        \vspace{5mm}\emph{Работу выполнил:}\\магистрант 2 года, гр. Z8430М\\
        Лебедев О. В.\\%
        \emph{Научный руководитель:}\\
        к.т.н. Коромысличенко В. Н.
    \end{flushright}
    
    \begin{center}
        \vspace{20mm}\small{Санкт-Петербург, 2021}
    \end{center}
\end{frame}


\section{Цели и задачи}
\begin{frame}{Цели и задачи}

    \textbf{Цель:}
        разработка программного обеспечения имитатора абонентов промышленной сети Modbus
        на основе онтологической модели критичного к ошибкам объекта контроля,
        унифицированного способа представления субъектно-ориентированного механизма изменения параметров,
        для имитации как внешнего окружения так и внутреннего состояния объекта
        контроля на основе онтологической модели необитаемого подводного аппарата.

    \textbf{Задачи}:
    \begin{enumerate}
        \item произвести декомпозицию модели объекта контроля предметной области для выделения метаклассов;
        \item создать онтологическую модель предметной области;
        \item создать и отладить программное обеспечение на основе онтологической модели;
        \item внедрить в жизненный цикл разработки программного обеспечения \textit{автоматизированных систем контроля и диагностики}
              вновь созданное программное обеспечение имитатора абонентов промышленной сети Modbus.
    \end{enumerate}
\end{frame}
\note{
    \todo{Не слишком ли большая цель?\ldots}

    Основная задача моего подразделения проектирование и разработка 
    аппаратно-программных комплексов проверки технического состояния объектов контроля критичных к ошибкам.
}

\begin{frame}{Практическая значимость}

    \textbf{Актуальность}:
    \begin{itemize}
        \item Риски с комплектующими и дополнительные \rouble.
        \item Прототипирование и отладка системы контроля (СК).
        \item Сокращение сроков разработки СК.
        \item Демонстрация заказчику ПО СК.
        \item Объект контроля (ОК) отсутствует.
        \item Дорогое время работы с ОК.
        \item Разрушение ОК на граничных значениях алгоритма.
        \item Риск для жизни работников.
    \end{itemize}
\end{frame}
