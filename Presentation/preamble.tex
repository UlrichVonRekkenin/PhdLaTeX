\begin{frame}[noframenumbering,plain]
    \setcounter{framenumber}{1}
    \maketitle
\end{frame}

\begin{frame}
    \frametitle{Положения, выносимые на защиту}

    {\textbf{Цель:}}
        разработка программного обеспечения имитатора абонентов промышленной сети Modbus 
        критичного к ошибкам объекта контроля.

  {\textbf{Задачи}}:
  \begin{enumerate}
    \item декомпозиция модели объекта контроля предметной области для выделения метаклассов;
    \item создать онтологическую модель предметной области;
  \end{enumerate}
  
\end{frame}
\note{
    Разработка программного обеспечения имитатора абонентов промышленной сети Modbus на основе онтологической модели
    критичного к ошибкам объекта контроля.

    {\textbf{Подзадачи}}:
    \begin{enumerate}
        \item создать и отладить программное обеспечение на основе онтологической модели;
        \item внедрить в жизненный цикл разработки программного обеспечения автоматизированных систем контроля и диагностики;
        \item вновь созданное программное обеспечение имитатора абонентов промышленной сети Modbus.
    \end{enumerate}
}

\begin{frame}
    \frametitle{Содержание}
    \tableofcontents
\end{frame}
\note{
    Работа состоит из трех глав.

    \medskip
    В первой главе \dots

    Во второй главе \dots

    Третья глава посвящена \dots
}
