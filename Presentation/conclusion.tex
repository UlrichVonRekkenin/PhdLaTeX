\section{Заключение}\subsection{Научная и практическая значимость. Перспективы}
\begin{frame}{Заключение}{Научная и практическая значимость. Перспективы}
    \textbf{Научная новизна}
    \begin{enumerate}
        \item Графоаналитический анализ модели АНПА для выявления общесистемных объектов.
        %
        \item Декомпозиция модели АНПА на основе результатов графоаналитического анализа с целью выделения метаклассов.
        %
        \item Создан программный комплекс имитатора абонентов сети Modbus.
    \end{enumerate}
    
    \textbf{Практическая значимость}
    \begin{enumerate}
        \item Апробация \leadingOrganizationTitle на предприятии.
        %
        \item Свидетельство о государственной регистрации программы для ЭВМ.
        %
        \item ПО имитатора использует те же сущности предметной области, что и ПО АСКД.
      \end{enumerate}
      
      \textbf{Перспективы}: 
      \begin{itemize}
        \item разработка \textit{тренажерного} или \textit{учебного} комплексов;
        \item приемо-сдаточные, пусконаладочные и предъявительские испытания ПО АСКД.
      \end{itemize}
\end{frame}


\subsection{Результаты}
\begin{frame}{Результаты}
    \textcolor{blue}{\tiny Лебедев, О. В. Применении парадигмы тестирования и паттернов проектирования при разработке специализированного программного
    обеспечения в ответственных областях применения [Текст] \/ О. В. Лебедев,
    В. Н. Коромысличенко \/\/ Гидроприбор. — 2019 — Т. 5, № 48 — С. 91—96.}
    \begin{figure}[h]
        \centering
        % \includegraphics[width=.4\linewidth]{registration.pdf}
        \fbox{\begin{minipage}[t]{0.4\linewidth}\includegraphics[width=\linewidth]{registration.pdf}\end{minipage}}
        \fbox{\begin{minipage}[t]{0.4\linewidth}\includegraphics[width=\linewidth]{implementation}\end{minipage}}
    \end{figure}
\end{frame}


\begin{frame}[plain, noframenumbering] % последний слайд без оформления
    \begin{center}
        \Huge
        Спасибо за внимание!
    \end{center}
\end{frame}
