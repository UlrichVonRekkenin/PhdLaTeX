\chapter{Мотивация}

https://www.viva64.com/ru/b/0438/

https://m.habr.com/ru/company/mailru/blog/370153/

http://www.osp.ru/os/1998/06/179592

Нужно ещё примеров \ldots

конечно ни одно программное неможет бытьпроверенно полностью (ссылка на авторитета).

\chapter{Протокол Modbus}\label{ch:ch1}

\section{Описание протокола}

\section{Обзор существующих решений}\label{sec:ch1/sec1}
\subsection{QSlave}

В свободном продукте\footnote{\url{https://github.com/maisvendoo/qslave}}, распространяемом по лиценции GPL используется
реализация имитатора множества конечных устройств, которые объеденены в одну сеть.
Этот программный продукт не подходит так как связь реализована по протоколу
Modbus/RTU, так как в конечном продукте необходима реализация через Modbus/TCP~IP.

\subsection{mEmulator}
Другой продукт\footnote{\url{http://ardsoft.ru/mEmulator.html}} требует исследования.


\subsection{Еще продукты}


\section{Выводы по существующим решениям}

Так как ни одно из решений не соответствует требованиям, придется реализовать свой
имитатор, архитектрура которого описана далее в главе \ref{sec:ch2}.