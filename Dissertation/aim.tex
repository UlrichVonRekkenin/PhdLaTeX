\chapter*{Назначение}

\todo{\textit{Органично вписать в повествование и расположжить в правильном месте \ldots}}

\section*{Описание}

Для проверки объекта контроля необходима система контроля.
Под ОК будем понимать \textit{АНПА} с телеуправлением,
предназначенный для исследования морских глубин и обнаружения аномалий.
Создание и отладка системы контроля сопряжена со следующими факторами.
%
\begin{itemize}
    \item Дороговизна и труднодоступность достаточного количества статистического материала
    натурных испытаний для отладки алгоритмов ПО СК экономически нецелесообразна, так же связана с риском для жизни испытующих.
    \item ОК находится на этапе проектирования.
    \item Отсутствует возможность подключения к ОК.
    \item Проверка граничных значений алгоритмов ОК, ведущих к разрушению ОК недопустима.
\end{itemize}

Так на этапах создания и отладки ПО СК необходимо иметь возможность проверки
функционирования алгоритмов проверки без непосредственного подключения к ОК.
Этого можно достичь путем создания имитатора ОК на базе ПЛК и системы аналогово-дискретных
модулей ввода-вывода.
В \todo{моих} текущих проектах стоимость подобного набора начинается от 500~тысяч рублей.
Следующие риски присущи этому подходу.
\begin{itemize}
    \item Увеличение накладных расходов (добавочная стоимость).
    \item Увеличение сроков разработки, так как невозможно начать непосредственную проверку до 
        создания, сборки и отладки имитатора на базе ПЛК.
    \item Риски, связанные со сроками поставки комплектующих имитатора на базе ПЛК.
\end{itemize}
Выбирая этот подход, возникает следующая проблема. При начале работы над новым объектом контроля
использовать уже имеющиеся комплектующие для создания другого имитатора на базе ПЛК
или же заказать новые --- см. выше риски.

При выборе первого сценария возможны следующие осложнения.
\begin{itemize}
    \item Придется частично или полностью разобрать имеющийся имитатор, тем самым при возникновении 
        осложнений с предыдущим ОК придется заново собирать его.
    \item Использовать аппаратную систему тумблеров для коммутации электрических сигналов,
        увеличивая сложность аппаратной составляющей имитатора на базе ПЛК.
    \item Использовать программные средства (if-else ветвления), что увеличит сложность программы имитатора.
    \item Человеческий фактор --- есть ненулевая вероятность перепутать тумблер или логическую ветвь в программе имитатора,
        тем самым вызвать недокументированное поведение и \todo{долго заниматься отладкой}.
\end{itemize}