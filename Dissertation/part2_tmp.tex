\chapter{Протокол Modbus}\label{ch:ch2}

\todo{%
    Должно быть не стоит привязываться к конкретному протоколу,%
    хотя он у меня один из доминирующих и имитатор именно этого протокола и сделан.%
    Имитация работы платы по ГОСТ 52070-2003 еще в работе...%
}

% алтернатива унифицированного 
% https://m.habr.com/ru/post/489314/?utm_source=vk&utm_medium=social&utm_campaign=chetvertushka-ethernet-a-staraya-skorost

\section{Описание протокола}

\section{Обзор существующих решений}\label{sec:ch2/sec1}

\todo{
    Наверно, надо показать почему я горожу свое решение...
}

\textbf{QSlave.}
В свободном продукте\footnote{\url{https://github.com/maisvendoo/qslave}}, распространяемом по лиценции GPL используется
реализация имитатора множества конечных устройств, которые объеденены в одну сеть.
Этот программный продукт не подходит так как связь реализована по протоколу
Modbus/RTU, так как в конечном продукте необходима реализация через Modbus/TCP~IP.

\textbf{mEmulator.}
Другой продукт\footnote{\url{http://ardsoft.ru/mEmulator.html}} требует исследования.


\textbf{Решение для всего мира от КТИ вычислительной техники СО РАН, Новосибирск, Россия.}
\cite{journal:vechisl_tech:2004_okolnischnikov} \ldots


\section*{Выводы}

\begin{itemize}
    \item Дорого
    \item Не учитывает моих особенностей
    \item Коллаборация с другими фирмами --- долго
    \item \ldots
\end{itemize}


Так как ни одно из решений не соответствует требованиям, придется реализовать свой
имитатор \ldots
