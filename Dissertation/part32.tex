\chapter{Описание АСУ ТП для систем контроля}\label{ch:chn}

\section{Введение\ldots}

% https://m.studref.com/495961/ekonomika/klassifikatsiya_imitatsionnyh_modeley_promyshlennogo_predpriyatiya

Автоматизированная система управления технологическим процессом (АСУ ТП) --- это термин,
имеющий отношение к ЭВМ, обеспечивающим управление различными техническими процессами \cite{journal:iter_research_guriev_2016}.
Изначально системы АСУ~ТП использовались исключительно на производстве, но с развитием технологий
и из-за сходства технических процессов АСУ~ТП вышло за рамки управления сугубо производственными процессами
и перешла в другие сферы деятельности, от управления транспортом %\cite{}
до управления техническими процессами зданий \cite{journal:vechisl_tech:2013:Golushko}
и более сложными техническими сооружениями (например, здесь \cite{journal:vechisl_tech:2004_okolnischnikov}).
%
Далее будет показано создание \todo{аналитической} модели объекта контроля
при создании имитатора объекта контроля и его применение на всех этапах жизненного цикла
объекта контроля: от постановки задачи и проектирования системы контроля, до пусконаладочных работ
и сдачи в опытную эксплуатацию.
