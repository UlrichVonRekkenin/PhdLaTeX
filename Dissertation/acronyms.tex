\chapter*{Список сокращений и условных обозначений} % Заголовок
\addcontentsline{toc}{chapter}{Список сокращений и условных обозначений}  % Добавляем его в оглавление
\noindent
% \begin{longtabu} to \dimexpr \textwidth-5\tabcolsep {r X}
\begin{longtabu} to \textwidth {r X}
    $\mathcal{M}$ & Множество дискретных и аналоговых параметров, передаваемых по сети Modbus \\
    %
    %
    \textbf{АНПА} & Автономный необитаемый подводный аппарат \\
    \textbf{АСУ~ТП} & Автоматизированная система управления технологическим процессом \\
    \textbf{БГР} & Блок гальванической развязки \\
    \textbf{ВУ} & Верхний уровень \\
    \textbf{НУ} & Нижний уровень \\
    \textbf{ОК} & Объект контроля \\
    \textbf{ПЛК} & Программируемый логический контроллер \\
    \textbf{ПО} & Программное обеспечение \\
    \textbf{СИ} & Синхроимпульс \\
    \textbf{СК} & Система контроля \\
    \textbf{ЭВМ} & Электронно-вычислительная машина \\
    \textbf{RDF} & Resource description framework \\
\end{longtabu}
\addtocounter{table}{-1}% Нужно откатить на единицу счетчик номеров таблиц, так как предыдующая таблица сделана для удобства представления информации по ГОСТ
