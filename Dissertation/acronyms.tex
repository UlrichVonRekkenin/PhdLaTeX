\chapter*{Список сокращений и условных обозначений} % Заголовок
\addcontentsline{toc}{chapter}{Список сокращений и условных обозначений}  % Добавляем его в оглавление
\noindent
% \begin{longtabu} to \dimexpr \textwidth-5\tabcolsep {r X}
\begin{longtabu} to \textwidth {r X}
    $\mathcal{M}$ & Множество дискретных и аналоговых параметров, передаваемых по сети Modbus \\
    %
    %
    \textbf{АКБ} & Аккумуляторная батарея \\
    \textbf{АНПА} & Автономный необитаемый подводный аппарат \\
    \textbf{АСУ~ТП} & Автоматизированная система управления технологическим процессом \\
    \textbf{БГР} & Блок гальванической развязки \\
    \textbf{БУБ} & Блок управления АКБ \\
    \textbf{БУД} & Блок управления двигателем \\
    \textbf{ВУ} & Верхний уровень \\
    \textbf{ГО} & Головной отсек \\
    \textbf{Д} & Движитель \\
    \textbf{НУ} & Нижний уровень \\
    \textbf{ОК} & Объект контроля \\
    \textbf{ПЛК} & Программируемый логический контроллер \\
    \textbf{ПО} & Программное обеспечение \\
    \textbf{ПУ} & Прибор управления \\
    \textbf{РМ} & Рулевая машинка \\
    \textbf{РП} & Рулевой привод \\
    \textbf{СИ} & Синхроимпульс \\
    \textbf{СК} & Система контроля \\
    \textbf{СПО} & Система поиска и обнаружения \\
    \textbf{СУД} & Система управления двигателем \\
    \textbf{СЧ} & Составная часть \\
    \textbf{ЭВМ} & Электронно-вычислительная машина \\
    \textbf{RDF} & Resource description framework \\
\end{longtabu}
\addtocounter{table}{-1}% Нужно откатить на единицу счетчик номеров таблиц, так как предыдующая таблица сделана для удобства представления информации по ГОСТ
