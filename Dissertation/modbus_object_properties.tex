\begin{landscape}

\begin{longtable}{|l|m{.15\textwidth}|l|m{.35\textwidth}|m{.3\textwidth}|}
\caption{Свойства объектов и типов компонентов имитатора сети Modbus.} \label{tbl:modbus_object_properties}\\
\hline
    \textbf{Название} & \textbf{Откуда действует} & \textbf{Куда действует} & \textbf{Назначение} & \textbf{Примечание} \\\hline
\endhead
%
\multicolumn{5}{|c|}{Объекты} \\\hline
%
\texttt{ИмеетЗначение} & \texttt{ModbusEle\-ment} & \texttt{ModbusData} & Связывает переменную Modbus с ее значением & \textit{функционально} \\\hline
\texttt{ИмеетУсловие} & \texttt{ModbusEle\-ment\-Writer} & \texttt{ModbusData} & Определяет, какая переменная отвечает за изменение значения & Обратно к \texttt{ЯвляетсяУсловием} \\\hline
\texttt{ЯвляетсяУсловием} & \texttt{ModbusData} & \texttt{ModbusElementWriter} & Показывает, что переменная отвечает за изменение состояния другой переменной &  \\\hline
\texttt{СоотнесенС} & \texttt{АНПА} & \texttt{ModbusElement} & Ставит в соответствие \textbf{объекту} значение переменной Modbus & 
    С \textbf{объектом} $i$ соотнесено средство измерения типа амперметр,
    единицы измерения \textit{А},
    диапазон измеряемой величины $[0, 550]$,
    десятичная приставка \textit{м} \\\hline
%
\end{longtable}
    
\end{landscape}