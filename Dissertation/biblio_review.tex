\section{Обзор литературы}

В статье \cite[journal:vechisl_tech:2004_okolnischnikov] описывается методология использования
имитационной модели на всех этапах жизненного цикла системы контроля.
Рассматривается применение этой методологии к созданию системы контроля для
Северомуйского тоннеля Байкал-Амурской железной дороги.

В пособии \cite[book:Immitmodelsiste] излагается расширенное содержание лекционной части учебной
дисциплины, посвященной моделированию и проектированию систем и входящей в учебные планы многих вузов. Рассматриваются общие определения
моделей и моделирования, раскрываются сущность и основные аспекты имитационного моделирования, приемы моделирования случайных величин, событий
и процессов, планирования, проведения компьютерных экспериментов, а также
наиболее употребительные методы обработки их результатов, имитационное
распределенное и мультиагентное моделирование. 

В монографии \cite[book:OhtilevSokolovUsupov] предлагается новый подход к формализации и решению проблемы комплексной автоматизации процессов мониторинга и управления состояниями сложных технических объектов (СТО), базирующийся на их полимодельном многокритериальном описании, концепциях и принципах теорий управления структурной динамикой, распознавания образов, недоопределенных вычислений и программирования в ограничениях. Описаны комбинированные методы и алгоритмы автоматического (автоматизированного) синтеза программ мониторинга и управления СТО при наличии некорректной, неточной и противоречивой измерительной информации. Приведены примеры практической реализации разработанного подхода применительно к СТО в критических приложениях (атомная энергетика, космонавтика и т.п.).
