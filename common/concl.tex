%% Согласно ГОСТ Р 7.0.11-2011:
%% 5.3.3 В заключении диссертации излагают итоги выполненного исследования, рекомендации, перспективы дальнейшей разработки темы.
%% 9.2.3 В заключении автореферата диссертации излагают итоги данного исследования, рекомендации и перспективы дальнейшей разработки темы.
\begin{enumerate}
  \item Был произведен графоаналитический анализ модели автономного необитаемого подводного аппарата для выявления
      общесистемных объектов.
  %
  \item Была произведена декомпозиция модели автономного необитаемого подводного аппарата на основе результатов
      графоаналитического анализа с целью выделения метаклассов.
  %
  \item Для выполнения поставленных задач был создан программный комплекс имитатора абонентов сети Modbus.
  %
  \item На предприятии \leadingOrganizationTitle была произведена апробация результатов разработки программного комплекса,
      что подтверждается соответствующим актом о внедрении.
  %
  \item Было получено свидетельство о государственной регистрации программы для ЭВМ.
  %
  \item Вновь созданное специализированное программное обеспечение имитатора 
        в своей работе использует те же сущности предметной области, что и программное обеспечение 
        автоматизированных систем контроля и диагностики.
\end{enumerate}

\textbf{Перспективой} развития этой работы является:
\begin{itemize}
    \item разработка \textit{тренажерного} или \textit{учебного} комплексов, которые могут быть использованы при обучении
            личного состава работе с объектом контроля с применением автоматизированных средств контроля и диагностики (АСКД);
    \item результаты этой работы будут положены в основу разработки АСКД следующего поколения;
    \item использование разработанного имитатора при приемо-сдаточных и пусконаладочных работах АСКД.
\end{itemize}

