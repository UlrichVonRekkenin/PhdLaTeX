{\aim} данной работы является разработка программного обеспечения 
  имитатора абонентов промышленной сети Modbus на основе онтологической модели критичного к ошибкам объекта контроля.

Для~достижения поставленной цели необходимо было решить следующие {\tasks}:
\begin{enumerate}
  \item произвести декомпозицию модели объекта контроля предметной области для выделения метаклассов;
  \item создать онтологическую модель предметной области;
  \item создать и отладить программное обеспечение на основе онтологической модели;
  \item внедрить в жизненный цикл разработки программного обеспечения автоматизированных систем контроля и диагностики
        вновь созданное программное обеспечение имитатора абонентов промышленной сети Modbus.
\end{enumerate}


{\novelty}
\begin{enumerate}
  \item была разработана система метаклассов с использованием экспертной системы построения онтологий и баз знаний \protege;
  \item было создано оригинальное программное обеспечение, основанное на событийно-ориентированных сценариях поведения;
  \item получено свидетельство о государственной регистрации программы для ЭВМ в федеральной службе по интеллектуальной собственности;
  \item произведено внедрение программного обеспечения на предприятии~\leadingOrganizationTitle.
\end{enumerate}

{\influence} в результате внедрения вновь созданного программного обеспечения:
\begin{itemize}
  \item упрощен этап прототипирования разрабатываемых автоматизированных систем контроля и диагностики объектов критичных к ошибкам;
  \item были существенно сокращены сроки проектирования, разработки и отладки программного обеспечения автоматизированных систем контроля и диагностики.
\end{itemize}

\todo{%
  {\methods}%
  \begin{enumerate*}[label=\arabic*\upshape)]
    \item В данной работе используется графоаналилитический метод для выявления общесистемных сущностей.
    \item Для выявленных сущностей строится онтология: определяются метаклассы, словарь.
    \item \ldots
  \end{enumerate*}
}

\todo{
  {\defpositions}
  \begin{enumerate}
    \item Использование онтологии позволяет отслеживать множество семантических связей и облегчает проектировать сложных программных комплексов.
    \item Использование онтологического анализа позволило создать субъектно-ориентированный событийный имитатор абонентов сети Modbus.
  \end{enumerate}
}

{\reliability} полученных результатов обеспечивается
  свидетельством о регистрации (рисунок \ref{app:fig:registration} приложения \ref{app:sec:registration}),
  актом о внедрении (рисунок \ref{app:fig:implementation} приложения \ref{app:sec:implementation}),  
  публикацией \cite{bib:my:ttd_with_patterns_2019}.


\todo{{\probation} Основные результаты работы внедрены~на предприятии \leadingOrganizationTitle.}

{\contribution} Автор принимал активное участие на всех этапах разработки программного обеспечения имитатора.
